\input{./setting.tex}
\begin{document}
% \maketitle
\pagestyle{fancy}
\section{目的}
太陽電池の各試験を行い,太陽電池の特性を知り,取り扱い上の要点を習得する.\\

\section{理論}
\subsection{再生可能エネルギー}
 太陽光,風力,その他非化石エネルギー源のうち,エネルギー源として永続的に利用することができると認められるもののこと.\\
 再生可能エネルギーとして,太陽光,風力,水力,地熱,太陽熱,大気中の熱その他の自然界に存する熱,バイオマスが挙げられる.
\footnote{環境省 平成26年度2050年再生可能エネルギー等分散型エネルギー普及可能性検証検討委託業務報告書 第1章再生可能エネルギー導入加速化の必要性,https://www.env.go.jp/earth/report/h27-01/,2019-7-1閲覧}

\subsection{太陽光発電の原理}
 現在最も多く使われている太陽電池は,シリコン系太陽電池である.この太陽電池では,電気的な性質の異なる2種類(p型,n型)の半導体を重ね合わせた構造をしている.\\
 太陽電池に太陽の光が当たると,電子と正孔が発生し,正孔はp型半導体へ,電子はn型半導体側へ引き寄せられる.このため、表面と裏面につけた電極に導線をつなげば,電子がn型からp型に,正孔はp型からn型に流れ,電流を取り出すことができる.
\footnote{太陽電池とは - 太陽光発電協会,http://www.jpea.gr.jp/knowledge/solarbattery/index.html,2019-7-1閲覧}
\begin{figure}[H]
  \centering
  \includegraphics[width=13cm]{./fig/fig01.png}
  \caption{太陽光発電の原理}
\end{figure}

\subsection{種類}
太陽光発電の種類は,使用している材料によって細かく分けられているが,大別すると図.2のようになる.\\
\begin{figure}[H]
  \centering
  \includegraphics[height=7cm]{./fig/fig02.png}
  \caption{太陽光発電の種類}
\end{figure}

\section{実験装置回路}
\begin{figure}[H]
  \centering
  \includegraphics[height=7cm]{./fig/fig03.png}
  \caption{実験装置回路}
\end{figure}

\section{使用機器}
太陽電池実験装置\\
照度計\\

\section{実験方法}
\bf{測定上の注意}\\
 実験装置のセレクトスイッチは以下の特性を持っているので,測定の際はすばやく読み取ること.\\
 設定:太陽電池がセットされていなくても,「ON」にしたときランプが約 30 秒点灯する. \\
 測定:太陽電池がセットされている場合に限り,約5秒点灯する.\\

\subsection{開放電圧の照度依存性試験}
\begin{enumerate}
  \item 実験装置のコンセントを差し込む前に以下の設定を行う.\\
  ・負荷スイッチは「OFF」にする.\\
  ・ スライドトランスは「0」にする.
  \item 照度計を太陽電池脇のほぼ中心にセットする.以降,照度計は極力動かさないこと.
  \item セレクトスイッチを設定にセット,装置の照明を ON にすることで,照度の設定ができる.100lx が理想だが,実験室の原明を感知するときがあるので,その時は最低値に設定する.
  \item セレクトスイッチを測定にセット,装置の照明を ON にすることで,各数値を読むことができる.この項目では発生電圧を読み取る.
  \item 照度を対数的に上げていき同様の測定を行う(最高照度は 20000lx).
\end{enumerate}

\subsection{短絡電流の照度依存性試験}
\begin{enumerate}
  \item 以下の設定を行う.\\
  ・ 負荷スイッチは「ON」にする.\\
  ・負荷抵抗は「100\%」にする.\\
  ・ スライドトランスは「0」にする.
  \item 照度の設定は,5.1 と同様に行い,発電電流を読み取る.
\end{enumerate}

\subsection{電圧電流特性の照度依存性試験}
\begin{enumerate}
  \item 以下の設定を行う.\\
  ・ 負荷スイッチは「ON」にする.\\
  ・ 負荷抵抗は「0\%」にする.\\
  ・スライドトランスは「0」にする.
  \item 照度の設定は,5.1 を参照.
  \item 一定限度のもと,負荷抵抗を 0\% から 100\%まで増加し,それぞれの発電電圧および発電電流を読み取る.
\end{enumerate}

\newpage
\section{結果}
\subsection{開放電圧の照度依存性試験}
 測定結果を表1に示す.また,グラフを短絡電流の照度依存特性と共に図4に示す.\\

\begin{table}[H]
  \centering
  \caption{解放電圧の照度依存特性}
  \begin{tabular}{@{}rrr@{}}
  \toprule
  \multicolumn{1}{c}{照度(目標値){[}lx{]}} & \multicolumn{1}{c}{照度(実測値){[}lx{]}} & \multicolumn{1}{c}{発生電圧{[}V{]}} \\ \midrule
  100    & 1.01E+02 & 11.0\\
  200   & 2.01E+02 & 13.7\\
  300   & 3.04E+02 & 14.7\\
  400   & 4.02E+02 & 15.4\\
  500   & 5.03E+02 & 15.8\\
  600   & 5.98E+02 & 16.1\\
  700   & 7.11E+02 & 16.4\\
  800   & 7.98E+02 & 16.6\\
  900   & 8.96E+02 & 16.7\\
  1000  & 1.05E+03 & 16.9\\
  2000  & 2.10E+03 & 17.9\\
  3000  & 2.99E+03 & 18.2\\
  4000  & 3.97E+03 & 18.5\\
  5000  & 5.05E+03 & 18.7\\
  6000  & 5.97E+03 & 18.8\\
  7000  & 6.97E+03 & 18.9\\
  8000  & 7.96E+03 & 19.0\\
  9000  & 9.06E+03 & 19.1\\
  10000 & 1.07E+04 & 19.1\\
  20000 & 2.05E+04 & 19.5\\
  最大値 & 2.55E+04 & 19.6\\ \bottomrule
\end{tabular}
\end{table}

\newpage
\subsection{短絡電流の照度依存性試験}
 測定結果を表2に示す.また,グラフを解放電圧の照度依存特性と共に図4に示す.\\

\begin{table}[H]
  \centering
  \caption{短絡電流の照度依存特性}
  \begin{tabular}{rrr}
\toprule
\multicolumn{1}{c}{照度(目標値)[lx]} & \multicolumn{1}{c}{照度(実測値)[lx]} & \multicolumn{1}{l}{発電電流[mA]} \\
\midrule
100   & 103   & 3 \\
200   & 201   & 9 \\
300   & 296   & 13 \\
400   & 400   & 18 \\
500   & 492   & 21 \\
600   & 597   & 25 \\
700   & 709   & 29 \\
800   & 798   & 32 \\
900   & 902   & 35 \\
1000  & 997   & 37 \\
2000  & 1970  & 63 \\
3000  & 2970  & 85 \\
4000  & 4010  & 106 \\
5000  & 5090  & 127 \\
6000  & 6050  & 143 \\
7000  & 7000  & 158 \\
8000  & 7980  & 174 \\
9000  & 9030  & 191 \\
10000 & 10000 & 205 \\
20000 & 20200 & 322 \\
最大値   & 25100 & 379 \\
\bottomrule
\end{tabular}

\end{table}


\subsection{電圧電流特性の照度依存性試験}

\begin{landscape}
  \begin{table}[htbp]
    \begin{center}
      \begin{tabular}{ccc}

        \begin{minipage}{.3\textheight}
          \begin{center}
            \caption{論理積(AND)}
            \begin{tabular}{rrrrrrrrr}
\toprule
\multicolumn{1}{l}{励磁電流I0[mA]} & \multicolumn{1}{l}{誘導起電力Eo[V]} & \multicolumn{1}{l}{損失電力Wo[W]} & \multicolumn{1}{l}{磁化電流Im[A]} & \multicolumn{1}{l}{磁界最大値Hm[A/m]} & \multicolumn{1}{l}{最大磁束密度Bm[T]} & \multicolumn{1}{l}{透磁率μe[H/m]} & \multicolumn{1}{l}{比透磁率μs} & \multicolumn{1}{l}{鉄損Wi[W/kg]} \\
\midrule
298   & 48.5  & 3     & 0.292 & 327.9316255 & 0.118894405 & 0.000362559 & 288.5149 & 1.10296301 \\
270.5 & 47.9  & 2.9   & 0.264 & 296.5772447 & 0.117423546 & 0.000395929 & 315.07033 & 1.064402825 \\
241.6 & 47.4  & 2.8   & 0.234 & 263.5370819 & 0.11619783 & 0.000440916 & 350.87015 & 1.024953482 \\
209.3 & 46.7  & 2.7   & 0.201 & 226.2891832 & 0.114481829 & 0.000505909 & 402.58992 & 0.987093978 \\
179.5 & 45.9  & 2.6   & 0.170 & 191.6091581 & 0.112520684 & 0.000587241 & 467.31125 & 0.949950863 \\
150   & 44.9  & 2.4   & 0.140 & 157.6641128 & 0.110069253 & 0.000698125 & 555.55019 & 0.870659515 \\
120   & 43.2  & 2.2   & 0.109 & 122.2339573 & 0.10590182 & 0.000866386 & 689.44827 & 0.796513769 \\
90    & 40    & 1.8   & 0.078 & 87.68059875 & 0.098057241 & 0.001118346 & 889.95142 & 0.645316588 \\
60    & 33.19 & 1.2   & 0.048 & 53.86563161 & 0.081362996 & 0.001510481 & 1202.0023 & 0.427164932 \\
45    & 26.87 & 0.7   & 0.037 & 41.27675991 & 0.065869952 & 0.001595812 & 1269.9069 & 0.242250314 \\
30    & 17.77 & 0.3   & 0.025 & 27.89733014 & 0.043561929 & 0.001561509 & 1242.6093 & 0.103267854 \\
15    & 6.62  & 0.02  & 0.015 & 16.52833824 & 0.016228473 & 0.000981858 & 781.3374 & 0.004897409 \\
0.03  & 0.022 & 0     & 0.000 & 0.033748278 & 5.39315E-05 & 0.001598051 & 1271.6889 & -4.22329E-08 \\
\bottomrule
\end{tabular}
          \end{center}
        \end{minipage}

        \begin{minipage}{.3\textheight}
          \begin{center}
            \caption{論理和(OR)}
            \begin{tabular}{rrrrrrrrr}
\toprule
\multicolumn{1}{l}{励磁電流I0[mA]} & \multicolumn{1}{l}{誘導起電力Eo[V]} & \multicolumn{1}{l}{損失電力Wo[W]} & \multicolumn{1}{l}{磁化電流Im[A]} & \multicolumn{1}{l}{磁界最大値Hm[A/m]} & \multicolumn{1}{l}{最大磁束密度Bm[T]} & \multicolumn{1}{l}{透磁率μe[H/m]} & \multicolumn{1}{l}{比透磁率μs} & \multicolumn{1}{l}{鉄損Wi[W/kg]} \\
\midrule
300   & 58    & 4     & 0.292 & 328.4442461 & 0.118485833 & 0.000360749 & 287.07469 & 1.450751613 \\
270.1 & 57.4  & 3.82  & 0.262 & 294.47946 & 0.117260117 & 0.000398195 & 316.87316 & 1.37830042 \\
240.3 & 56.8  & 3.66  & 0.231 & 260.4236352 & 0.116034402 & 0.00044556 & 354.56553 & 1.314507839 \\
210.2 & 56    & 3.58  & 0.200 & 225.261654 & 0.114400114 & 0.000507854 & 404.13766 & 1.28749625 \\
180.1 & 55    & 3.4   & 0.169 & 190.2933521 & 0.112357255 & 0.000590442 & 469.8591 & 1.21868895 \\
150   & 53.6  & 3.16  & 0.138 & 155.161689 & 0.109497252 & 0.000705698 & 561.57641 & 1.127298406 \\
120.2 & 51.6  & 2.82  & 0.107 & 120.433449 & 0.105411534 & 0.000875268 & 696.51608 & 0.997393051 \\
90    & 47.4  & 2.3   & 0.076 & 85.26959887 & 0.096831525 & 0.001135593 & 903.67588 & 0.806917549 \\
60    & 38.62 & 1.42  & 0.047 & 53.33789828 & 0.078895222 & 0.001479159 & 1177.0772 & 0.489076387 \\
45    & 30.84 & 0.9   & 0.034 & 38.53416229 & 0.063001777 & 0.001634959 & 1301.0591 & 0.309473125 \\
30    & 20.01 & 0.38  & 0.023 & 26.12582581 & 0.040877612 & 0.001564644 & 1245.104 & 0.130769205 \\
15    & 7.46  & 0.02  & 0.015 & 16.60243129 & 0.01523973 & 0.000917922 & 730.45876 & 0.003865391 \\
0.02  & 0.023 & 0     & 0.000 & 0.022498852 & 4.69858E-05 & 0.002088363 & 1661.8662 & -4.61595E-08 \\
\bottomrule
\end{tabular}
          \end{center}
        \end{minipage}

        \begin{minipage}{.3\textheight}
          \begin{center}
            \caption{論理和(OR)}
            \begin{tabular}{@{}rrrr@{}}
  \toprule
  \multicolumn{1}{c}{負荷抵抗} & \multicolumn{1}{c}{発電電圧{[}V{]}} & \multicolumn{1}{c}{発電電流{[}mA{]}} & \multicolumn{1}{c}{電力{[}mW{]}} \\ \midrule
  0   & 19.1 & 90  & 1719.0\\
  5   & 19.0 & 94  & 1786.0\\
  10  & 18.9 & 99  & 1871.1\\
  15  & 18.9 & 105 & 1984.5\\
  20  & 18.8 & 111 & 2086.8\\
  25  & 18.7 & 117 & 2187.9\\
  30  & 18.6 & 130 & 2418.0\\
  35  & 18.5 & 141 & 2608.5\\
  40  & 18.4 & 153 & 2815.2\\
  45  & 18.2 & 164 & 2984.8\\
  50  & 18.0 & 184 & 3312.0\\
  55  & 17.8 & 203 & 3613.4\\
  60  & 17.5 & 225 & 3937.5\\
  65  & 17.0 & 249 & 4233.0\\
  70  & 15.8 & 281 & 4439.8\\
  75  & 13.4 & 291 & 3899.4\\
  80  & 10.2 & 298 & 3039.6\\
  85  & 7.4  & 305 & 2257.0\\
  90  & 5.1  & 309 & 1575.9\\
  95  & 1.9  & 315 & 598.5 \\
  100 & 0.1  & 318 & 31.8\\ \bottomrule
\end{tabular}
          \end{center}
        \end{minipage}

      \end{tabular}
    \end{center}
  \end{table}

\end{landscape}

\end{document}
