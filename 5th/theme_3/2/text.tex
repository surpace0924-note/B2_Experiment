\input{./setting.tex}
\begin{document}
% \maketitle
\pagestyle{fancy}
\section{目的}
 立体導波管回路におけるマイクロ波の周波数測定法と定在波測定法の動作原理を理解する.

\section{予習}
マイクロ波回路用の空洞周波数計について構造や動作原理を調べよ.\\
 3ギガヘルツ以上のマイクロ波回路の周波数測定に用いられる装置.原理的には,寸法の定まった金属箱が,寸法に応じて定まる特定の波長のマイクロ波と共振する空胴共振器を利用して,波長の逆数としての周波数を測定する.\\

マイクロ波回路における,整合,反射係数,定在波,定在波比について調べよ.\\
・整合:信号源インピーダンスと負荷インピーダンスと伝送路の特性インピーダンスが等しくなった状態のこと\\
・反射係数:進行する波に対し反射して戻ってくる波の電圧振幅の割合のこと.\\
・定在波:周期,速さ,振幅が同一で逆方向に進行する波が重なると,波がその場で振動するように見える現象が生じる.この波のことを,定在波,あるいは定常波と呼ぶ.\\
・定在波比:定在波の最大の振幅と最小の振幅の比率.\\

\section{理論}
 高周波伝送には一般に同軸ケーブルが使用されるが,マイクロ波の周波数では表皮効果による損失が大きいため断面寸法比が2:1の方形導波管が使用される.\\
導波管による立体マイクロ波回路で金属製の様々な形状の回路素子があり,空洞による共振回路や金属板による減衰器,フィルタ,分岐や合成を行う部品等がある.\\
 高周波であるため分布定数回路として回路解析をする必要があり,特性インピーダンスの整合が取れない場合は反射はによる定在波が発生する.また,導波管での伝送は3次元的となり,通行の自由空間での電磁波の波長より長い管内波長をもつ.\\

導波管の遮断周波数 $f_c=c/\lambda c$\\
遮断波長$\lambda c = 2a$\\
管内波長 $\lambda_g = \dfrac{\lambda}{\sqrt{1-(\frac{\lambda}{\lambda_c})^2}}$\\
定在波比 $S=V_{max}/V_{min}=(1+\gamma)/(1-\gamma)$\\

\section{実験}
1) 測定器にクリスタルマウントをクリップで接続する.マイクロ波発振器専用電源のMODSELECTORをCWにし,投入後,電源電圧計が9Vであることを確認する.\\
クリスタルマウント出力をuVメータで観測し,マイクロ波の存在を確認する.\\

2) 空洞周波数計のつまみを回転させ,uVメーターの表示の最小点を探索せよ.ツマミの数値を読み校正周波数構成図から発信周波数を求めよ.\\

3) 2)の測定周波数から自由空間のマイクロ波波長を計算で求める.次にノギスを用いて実験装置の導波管の長辺方向の内側寸法を計測し,遮断波長を求める。これらの測定値から管内波長を計算で求める.\\

4) 定在波測定装置の探針を移動しながら探針のクリスタルマウント出力をuVメーターで測定し,グラフにプロットする.そのグラフから定在波の最大値間の距離Lを求める。\\

5) 4)のVmaxとVminの値から,電圧定在波比を求め,さらに反射係数を計算で求めよ.\\

\end{document}
